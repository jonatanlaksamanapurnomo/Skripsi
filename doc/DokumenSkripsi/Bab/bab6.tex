\chapter{Kesimpulan dan Saran}
\label{chap:kesimpulan_dan_saran}
Bab ini membahas kesimpulan berdasarkan implementasi dan pengujian, serta saran-saran untuk pengembangan berikutnya.

\section{Kesimpulan}
Berdasarkan hasil penelitian yang dilakukan, diperoleh kesimpulan-kesimpulan sebagai berikut:
\begin{enumerate}
	\item Berdasarkan hasil perbandingan pengujian satu \ref{subsec:pengujian1} dengan pengujian dua \ref{subsec:pengujian2}. Dapat disimpulkan bahwa mayoritas pengguna aplikasi KIRI akan menggunakan aplikasinya pada saat jam kerja.
	
	\item Berdasarkan hasil pengujian tiga \ref{subsec:pengujian3} dan pengujian lima \ref{subsec:pengujian5}. Dapat disumpulkan bahwa titik yang paling sering dijadikan tempat awal pencarian rute terdapat di daerah Taman Sari, lebih detail nya terdapat pada tiga titik utama yaitu Paris Van Java, Institut Teknologi Bandung, McDonald's Simpang Dago.
	
	\item Berdasarkan hasil pengujian empat \ref{subsec:pengujian4}. Dapat disimpulkan terdapat tiga titik yang paling sering menjadi destinasi tujuan pencarian rute yaitu Pasar Baru Trade Center, Stasiun Kereta Bandung, Braga City Walk.
	
	\item Berdasarkan hasil pengujian enam \ref{subsec:pengujian6}. Dapat disimpulkan terdapat sebuah  titik yang paling sering menjadi destinasi tujuan pencarian rute pada saat weekend  yaitu Cafe Jurnal Risa.
	
	\item Berdasarkan hasil seluruh pengujian perangkat lunak ini telah berhasil memvisualisasikan data histori KIRI kedalam bentuk \textit{heat map} dan \textit{marker clustering} dengan menggunakan \textit{Google Maps}.
	
\end{enumerate}

\section{Saran}
Penulis memiliki beberapa saran untuk pengembangan aplikasi visualisasi data histori KIRI dengan \textit{Google Maps}:
\begin{enumerate}
	\item Perlu diperhatikan bahwa data log histori KIRI memiliki banyak potensi untuk dapat diolah, pada perangkat lunak ini hanya akan mengolah data berdasarkan tiga atribut yaitu start/finish, hari, dan waktu. Jika atribut dapat ditambah seperti atribut tanggal pastinya hasl kesimpulan akan dapat lebih baik. 
	
	\item Pengujian pada perangkat lunak ini hanya dengan menggunakan metode observasi. Pengujian dapat dilakukan dengan metode \textit{Exploratory data analysis} (EDA) memberikan hasil kesimpulan yang lebih baik.
\end{enumerate}