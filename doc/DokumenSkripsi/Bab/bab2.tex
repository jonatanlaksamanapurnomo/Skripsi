%versi 2 (8-10-2016)
\chapter{Landasan Teori}
\label{chap:teori}

\section{Kiri Website }
\label{sec:Kiri} 
Kiri adalah aplikasi navigasi angkutan umum berbasis web yang melayani Bandung dan kota-kota lain di Indonesia.\cite{pascal:17:kiri}.
Pada awal pembuatannya Kiri dibuat untuk tujuan komersial.Namun karena dinilai kurang sukses project kiri sekarang menjadi open source project yang dapat di akses di url: \url{https://projectkiri.id/}

\section{Penyimpanan Data}
\label{sec:penyimpanan data}
Penyimpanan data dapat dilakukan dengan beberapa tipe data, contohnya adalah CSV dan JSON.

\subsection{JSON}
\label{subsec:json}
JSON (\textit{JavaScript Object Notation}) adalah format pertukaran data yang ringan, mudah dibaca dan ditulis oleh manusia, serta mudah diterjemahkan dan dibuat oleh komputer \footnote{https:\slash \slash www.json.org}. JSON merupakan format teks yang tidak bergantung pada bahasa pemprograman apapun karena menggunakan gaya bahasa yang umum digunakan oleh bahasa pemrograman C termasuk C, C++, C\#, Java, JavaScript, Perl, Python, dan lain-lain. Oleh karena sifat-sifat tersebut, menjadikan JSON ideal sebagai bahasa pertukaran-data. JSON memiliki enam tipe data, yaitu \textit{string}, angka, \textit{null}, \textit{array} (ditandai dengan tanda kurung siku ($\left [ \right ]$)), \textit{object} (ditandai dengan tanda kurung kurawal (\{\})), dan \textit{boolean} (\textit{true} dan \textit{false}). Struktur utama JSON terdiri dari pasangan nama atau nilai yang dipisahkan dengan tanda titik dua (:). Contoh struktur JSON mengenai tipe dan jenis atribut dapat dilihat pada Listing~\ref{listing:JSON}.

\begin{lstlisting}[caption=Contoh Struktur JSON, label=listing:JSON]
 {
 "timestamp":"2014-1-2:0:11",
 "start":"-6.8972513,107.6385574",
 "finish":"-6.91358,107.62718"
 }
\end{lstlisting}

\subsection{CSV}
\label{subsec:csv}
CSV (\textit{Comma Separated Values}) adalah suatu format data dalam basis data di mana setiap nilai atribut dipisahkan dengan tanda koma (,) dan setiap baris data ditandai dengan baris baru. CSV digunakan untuk bertukar data dan mengonversi data dari sebuah program \textit{spreadsheet} ke program \textit{spreadsheet} lainnya \cite{RFC4180}. Contoh CSV dapat dilihat pada Listing~\ref{listing:CSV}.

\begin{lstlisting}[caption=Contoh CSV, label=listing:CSV]
    logId,APIKey,Timestamp (UTC),Action,AdditionalData
    113909,E5D9904F0A8B4F99,2/1/2014 0:07,PAGELOAD,/5.10.83.30/
    113910,E5D9904F0A8B4F99,2/1/2014 0:07,PAGELOAD,/5.10.83.49/
\end{lstlisting}
	
\section{Google Maps Javascript API}
\label{sec:googlemaps}
Google Maps adalah layanan pemetaan web yang dikembangkan oleh Google. Menawarkan citra satelit, foto udara, dan peta jalan yang interaktif, kondisi lalu lintas secara \textit{real time} \cite{mehta:19:gmaps}.
Google Maps dalam service nya telah menyediakan \textit{API (pplication programming interface)} yang dapat di gunakan untuk public.
aplication programming interface adalah \textit{computer interface} yang mengatur komunikasi antar perangkat lunak \cite{libby:20:api}.
Google Maps telah menyediakan beberapa teknik pemetaan data yaitu :
 \begin{itemize}
     \item \textit{Heat Map}
     \item \textit{Marker Clustering}

 \end{itemize}
 \subsection{\textit{Heat Map}}
 \label{subsec:heat map}
 \textit{Heat Map } adalah teknik visualisasi data dimana data akan di representasikan sebagai warna pada suatu space 2 dimensi, semakin banyak data yang terdapat pada suatu tempat maka intensitas warna yang di berikan akan semakin tinggi.
 
 \subsection{\textit{Marker Clustering}}
 \label{subsec:heat map}
 \textit{Marker Clustering } adalah teknik visualisasi data dimana data akan di representasikan sebagai tanda / \textit{Mark} pada suatu space 2 dimensi, semakin banyak data yang terdapat pada suatu tempat maka akan semakin banyak quantitas penanda / \textit{Mark} yang di berikan.

 
