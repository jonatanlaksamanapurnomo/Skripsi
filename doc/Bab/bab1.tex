%versi 2 (8-10-2016) 
\chapter{Pendahuluan}
\label{chap:intro}
   
\section{Latar Belakang}
\label{sec:label}

Kemajuan teknologi memudahkan manusia untuk mencari berbagai macam informasi. Salah satu informasi yang dapat diperoleh adalah informasi tentang navigasi transportasi publik. KIRI  adalah perangkat lunak yang berguna sebagai navigasi antar kota menggunakan transportasi publik dengan menggunakan perangkat peta digital\cite{pascal:17:KIRI}. Pada awal pembuatannya KIRI dibuat untuk tujuan komersial. Namun karena dinilai kurang sukses, projek KIRI sekarang menjadi open source projek yang dapat di akses. Aplikasi KIRI memiliki beberapa fitur sebagai berikut:
\begin{itemize}
	\item Pemilihan rute tercepat menggunakan angkutan kota.
	perangkat lunak KIRI dapat menentukan rute terbaik untuk berpegian dengan angkutan umum. 
	\begin{figure}[H]
	\centering
	\includegraphics[scale=0.3]{Gambar/kiri-example-1}
	\caption{Tampilan utama website KIRI}
	\label{fig:my_label}
\end{figure}
	\item Memiliki fitur multi bahasa.
	perangkat lunak kiri dilengkapi dengan fitur multi bahasa. 
\begin{figure}[H]
	\centering
	\includegraphics[scale=0.3]{Gambar/multibahasa.PNG}
	\caption{Multi Bahasa}
	\label{fig:my_label}
\end{figure}
	\item Dapat menampilkan instruksi lengkap  mencapai tujuan.
	perangkat lunak kiri juga menampilkan seluruh instruksi lengkap penggunaan angkutan umum. 
\end{itemize}
\begin{figure}[H]
	\centering
	\includegraphics[scale=0.3]{Gambar/instuksi.PNG}
	\caption{Instruksi KIRI}
	\label{fig:my_label}
\end{figure}



Pada perangkat lunak KIRI seluruh aktivitas yang dilakukan oleh user sudah tercatat. Data yang tercatat disebut juga dengan data histori. Data histori kiri memiliki jumlah record yang cukup banyak sehingga memungkinkan untuk mendapatkan informasi dari data tersebut. Tetapi data histori tersebut belum diolah secara maksimal. Data histori KIRI memiliki format \textit{comma separated values} (csv) yang memiliki lima atribut yaitu:
\begin{itemize}
    \item LogId
    Id unik  sebagai penanda satu record didalam data histori KIRI.
    \item APIKey
     atribut api key yang digunakan ketika melakukan perintah pada perangkat lunak KIRI.
    \item Timestamp (UTC)
    atribut untuk mencatat waktu perintah dilakukan format berbentuk \textit{timestamp}.
    \item Action
    jenis action yang dilakukan user pada saat menggunakan perangkat lunak KIRI. Terdapat empat nilai action pada data histori KIRI yakni:
    \begin{itemize}
        \item \textit{PAGELOAD}.
        \item \textit{SEARCHPLACE}.
        \item \textit{WIDGETLOAD}.
        \item \textit{FINDROUTE}.
    \end{itemize}
    \item AdditionalData atribut yang digunakan untuk mencatat informasi tambahan berdasarkan action yang dipilih. Nilai pada additionaldata akan bergantung pada action yang dipilih:
    \begin{itemize}
        \item jika action bernilai \textit{PAGELOAD} maka additionaldata akan bernilai ip dari pengakses.
        \item jika action bernilai \textit{SEARCHPLACE} maka additionaldata akan bernlai keyword yang dituliskan oleh pengakses.
        \item jika action bernilai \textit{FINDROUTE} maka additionaldata akan bernilai posisi tempat dan tujuan dalam bentuk langtitude dan longtitude  yang dicari oleh pengakses.
        \item jika action bernilai \textit{WIDGETLOAD} maka additionaldata akan bernilai alamat url dari penyedia widget.
    \end{itemize}
\end{itemize}

Visualisasi Data adalah teknik  untuk mengkomunikasikan data atau informasi dengan menggunakan objek visual  seperti \textit{graphic, chart, diagram, dll}. Salah satu objek visual yang dapat digunakan untuk merepresentasikan data adalah \textit{Google Maps}.

Metode yang akan digunakan dalam memvisualisasikan data adalah \textit{Heat Map} dan \textit{Marker Clustering}. \textit{Heat Map} adalah teknik visualisasi data yang menunjukkan besarnya suatu fenomena sebagai warna dalam dua dimensi. Sedangkan \textit{Marker Clustering}  adalah teknik visualisasi data  yang  mengelompokan \textit{marker} atau \textit{pointer} yang jarak \textit{latitude} dan \textit{longitude} nya saling berdekatan antara suatu \textit{marker} dengan marker yang lainnya.

Pada skripsi ini akan dibangun perangkat lunak yang dapat memvisualisasikan data histori KIRI. Perangkat lunak ini akan menggunakan metode visualisasi \textit{heat map} dan \textit{marker clustering} dari hasil visualisasi tersebut akan diambil suatu pola kesimpulan dari data histori KIRI.


\section{Rumusan Masalah}
\label{sec:rumusan}
Rumusan masalah dari topik ini adalah sebagai berikut:
\begin{itemize}
  \item Bagaimana memvisualisasikan data histori KIRI?
  \item Bagaimana menemukan pola dari data histori KIRI?
  \item Bagaimana penerapan metode \textit{heat map} pada visualisasi data histori KIRI?
  \item Bagaimana penerapan metode \textit{marker clustering} pada visualisasi data histori KIRI?


\end{itemize}

\section{Tujuan}
\label{sec:tujuan}
Tujuan dari topik ini adalah sebagai berikut:
\begin{itemize}
  \item Mempelajari \textit{Google Maps Javascript API}.
  \item Melakukan observasi data.
  \item Mengimplementasikan metode \textit{Heat map} pada visualisasi data histori KIRI.
  \item Mengimplementasikan metode \textit{Marker clustering} pada visualisasi data histori KIRI.


\end{itemize}

\section{Batasan Masalah}
\label{sec:batasan}
\dtext{9}

\section{Metodologi}
\label{sec:metlit}
Metodologi yang digunakan dalam penelitian ini adalah:
	\begin{enumerate}

		\item Mempelajari \textit{Google Maps Javascript API} khususnya \textit{Heat Map} dan \textit{Marker Clustering}.
		\item Analisis masalah perangkat lunak yang akan dibangun.
		\item Merancang perangkat lunak yang akan dibangun.
		\item Membangun perangkat lunak yang mengimplementasikan \textit{Heat Map} atau \textit{Marker Clustering} dengan memanfaatkan \textit{Google Maps Javascript API}.
		\item Menentukan pola dari hasil visualisasi data.
		\item Analisis hasil pengujian dan mengambil kesimpulan.
	\end{enumerate}


\section{Sistematika Pembahasan}
\label{sec:sispem}
Laporan penelitian tersusun ke dalam enam bab secara sistematis sebagai berikut.
\begin{itemize}
    \item Bab 1 Pendahuluan\\
    Berisi latar belakang, rumusan masalah, tujuan, batasan masalah, metodologi penelitian, dan sistematika pembahasan.
   
   \item Bab 2 Dasar Teori\\
    Berisi  metode penentuan pola, \textit{library Google Maps} dan bahasa pemograman \textit{Javascript}
   
    \item Bab 3 Analisis\\
    Berisi analisis masalah terkait implementasi \textit{Goole Map}, studi kasus teknik penentuan pola yang diimplementasikan, dan gambaran umum perangkat lunak yang meliputi diagram aktivitas dan diagram kelas.
  
    \item Bab 4 Perancangan Perangkat Lunak\\
    Berisi perancangan perangkat lunak yang akan dibangun, meliputi perancangan antarmuka,
diagram kelas lengkap dan masukan perangkat lunak.
    
    \item Bab 5 Implementasi dan Pengujian\\
Berisi implementasi antarmuka perangkat lunak, pengujian fungsional, pengujian eksperimental
serta kesimpulan dari pengujian.
   
    \item Bab 6 Kesimpulan dan Saran\\
    Berisi kesimpulan dari awal hingga akhir penelitian dan saran untuk penelitian berikutnya.
\end{itemize}