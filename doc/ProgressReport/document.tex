\documentclass[a4paper,twoside]{article}
\usepackage[T1]{fontenc}
\usepackage[bahasa]{babel}
\usepackage{graphicx}
\usepackage{graphics}
\usepackage{float}
\usepackage[cm]{fullpage}
\pagestyle{myheadings}
\usepackage{etoolbox}
\usepackage{setspace} 
\usepackage{lipsum} 
\setlength{\headsep}{30pt}
\usepackage[inner=2cm,outer=2.5cm,top=2.5cm,bottom=2cm]{geometry} %margin
% \pagestyle{empty}

\makeatletter
\renewcommand{\@maketitle} {\begin{center} {\LARGE \textbf{ \textsc{\@title}} \par} \bigskip {\large \textbf{\textsc{\@author}} }\end{center} }
\renewcommand{\thispagestyle}[1]{}
\markright{\textbf{\textsc{Laporan Perkembangan Pengerjaan Skripsi\textemdash Sem. Ganjil 2020/2021}}}

\onehalfspacing
 
\begin{document}

\title{\@judultopik}
\author{\nama \textendash \@npm} 

%ISILAH DATA BERIKUT INI:
\newcommand{\nama}{Jonathan Laksamana Purnomo}
\newcommand{\@npm}{2016730081}
\newcommand{\tanggal}{01/15/2021} %Tanggal pembuatan dokumen
\newcommand{\@judultopik}{Visualisasi Data Histori KIRI pada Google Maps} % Judul/topik anda
\newcommand{\kodetopik}{TAB0901}
\newcommand{\jumpemb}{2} % Jumlah pembimbing, 1 atau 2
\newcommand{\pembA}{Pascal Alfadian Nugroho}
\newcommand{\pembB}{-}
\newcommand{\semesterPertama}{45 - Ganjil 20/21} % semester pertama kali topik diambil, angka 1 dimulai dari sem Ganjil 96/97
\newcommand{\lamaSkripsi}{1} % Jumlah semester untuk mengerjakan skripsi s.d. dokumen ini dibuat
\newcommand{\kulPertama}{Skripsi 1} % Kuliah dimana topik ini diambil pertama kali
\newcommand{\tipePR}{B} % tipe progress report :
% A : dokumen pendukung untuk pengambilan ke-2 di Skripsi 1
% B : dokumen untuk reviewer pada presentasi dan review Skripsi 1
% C : dokumen pendukung untuk pengambilan ke-2 di Skripsi 2

% Dokumen hasil template ini harus dicetak bolak-balik !!!!

\maketitle

\pagenumbering{arabic}

\section{Data Skripsi} %TIDAK PERLU MENGUBAH BAGIAN INI !!!
Pembimbing utama/tunggal: {\bf \pembA}\\
% Pembimbing pendamping: {\bf \pembB}\\
Kode Topik : {\bf \kodetopik}\\
Topik ini sudah dikerjakan selama : {\bf \lamaSkripsi} semester\\
Pengambilan pertama kali topik ini pada : Semester {\bf \semesterPertama} \\
Pengambilan pertama kali topik ini di kuliah : {\bf \kulPertama} \\
Tipe Laporan : {\bf \tipePR} -
\ifdefstring{\tipePR}{A}{
			Dokumen pendukung untuk {\BF pengambilan ke-2 di Skripsi 1} }
		{
		\ifdefstring{\tipePR}{B} {
				Dokumen untuk reviewer pada presentasi dan {\bf review Skripsi 1}}
			{	Dokumen pendukung untuk {\bf pengambilan ke-2 di Skripsi 2}}
		}
		
\section{Latar Belakang}
Kemajuan teknologi memudahkan manusia untuk mencari berbagai macam informasi. Salah satu informasi yang dapat diperoleh adalah informasi tentang navigasi transportasi publik. KIRI  adalah perangkat lunak yang berguna sebagai navigasi antar kota menggunakan transportasi publik dengan menggunakan perangkat peta digital dan informasi posisi dengan menggunakan satelit GPS.

Visualisasi Data adalah teknik  untuk mengkomunikasikan data atau informasi dengan menggunakan objek visual  seperti \textit{graphic, chart, diagram, dll}. Salah satu objek visual yang dapat digunakan untuk merepresentasikan data adalah \textit{Google Maps}.

Pada perangkat lunak KIRI seluruh aktivitas yang dilakukan oleh user sudah tercatat. Data yang tercatat disebut juga dengan data histori. Tetapi data histori tersebut belum diolah secara maksimal. Visualisasi data  adalah metode yang akan digunakan untuk mengolah data histori sehingga dapat ditemukan pola tertentu.

Metode yang akan digunakan dalam memvisualisasikan data adalah \textit{Heat Map } dan \textit{Marker Clustering}. \textit{Heat Map} adalah teknik visualisasi data yang menunjukkan besarnya suatu fenomena sebagai warna dalam dua dimensi. Sedangkan \textit{Marker Clustering}  adalah teknik visualisasi data  yang  mengelompokan \textit{marker} atau \textit{pointer} yang jarak \textit{latitude} dan \textit{longitude} nya saling berdekatan antara suatu \textit{marker} dengan marker yang lainnya.
\section{Rumusan Masalah}
Rumusan masalah dari topik ini adalah sebagai berikut:
\begin{itemize}
  \item Bagaimana memvisualisasikan data histori KIRI?
  \item Bagaimana menemukan pola dari data histori KIRI?
  \item Bagaimana membangun perangkat lunak yang dapat memvisualisasikan data histori KIRI?

\end{itemize}


\section{Tujuan}
Tujuan dari topik ini adalah sebagai berikut:
\begin{itemize}
  \item Mempelajari teknik visualisasi data.
  \item Melakukan observasi data.
  \item Membangun perangkat lunak yang dapat memvisualisasikan data histori KIRI.

\end{itemize}


\section{Detail Perkembangan Pengerjaan Skripsi}
Detail bagian pekerjaan skripsi sesuai dengan rencan kerja/laporan perkembangan terkahir :
	\begin{enumerate}
		\item \textbf{Mempelajari atribut-atribut pada histori data KIRI.}\\
		{\bf Status :} Ada sejak rencana kerja skripsi.\\
		{\bf Hasil :} Proses pembelajaran sudah dilakukan sebanyak 2x. Proses pembelajaran yang pertama dilakukan saat mengerjakan prasyarat skripsi. Proses pembelajaran kedua dilakukan saat mempelajari \textit{Google Maps API}
		
		\item \textbf{Mempelajari visualisasi data.}\\
		{\bf Status :} Ada sejak rencana kerja skripsi.\\
		{\bf Hasil :} Proses pembelajaraan sudah dilakukan sebanyak 1x. Proses pembelajaran dilakukan pada mata kuliah Analisis Big Data yang diambil bersamaan dengan skripsi 1

		\item \textbf{Mempelajari \textit{Google Maps API}  khususnya \textit{HeatMap} dan \textit{MarkerClustering}}\\
		{\bf Status :} Ada sejak rencana kerja skripsi.\\
		{\bf Hasil :} Proses pembelajaran sudah dilakukan sebanyak 1x. Proses pembelajaran dilakukan dengan membuat prototype perangkat lunak yang dapat diakses pada halaman pada halaman website (https://github.com/jonatanlaksamanapurnomo/KiriHistory)

		\item \textbf{Membuat desain perangkat lunak}\\
		{\bf Status :} Ada sejak rencana kerja skripsi.\\
		{\bf Hasil :} Belum dikerjakan.

		\item \textbf{Mengimplementasi perangkat lunak}\\
		{\bf Status :} Ada sejak rencana kerja skripsi.\\
		{\bf Hasil :} Belum dikerjakan.

		\item \textbf{Menulis dokumen skripsi} \\
		{\bf Status :} Ada sejak rencana kerja skripsi.\\
		{\bf Hasil :} Belum dikerjakan.

	\end{enumerate}

\section{Pencapaian Rencana Kerja}
Langkah-langkah kerja yang berhasil diselesaikan dalam Skripsi 1 ini adalah sebagai berikut:
\begin{enumerate}
\item Mempelajari atribut-atribut pada histori data KIRI.
\item Mempelajari visualisasi data.
\item Mempelajari \textit{Google Maps API}  khususnya \textit{HeatMap} dan \textit{MarkerClustering}.
\end{enumerate}



% \section{Kendala yang Dihadapi}
% %TULISKAN BAGIAN INI JIKA DOKUMEN ANDA TIPE A ATAU C
% Kendala - kendala yang dihadapi selama mengerjakan skripsi :
% \begin{itemize}
% 	\item Sulit menemukan refrensi yang valid dan lengkap untuk beberapa \textit{tools} yang digunakan dalam skripsi ini.
% 	\item Terlalu
% \end{itemize}

\vspace{6cm}
\centering Bandung, \tanggal\\
\vspace{2cm} \nama \\ 
\vspace{1cm}

\ifdefstring{\jumpemb}{2}{
\vspace{1.5cm}
\begin{centering} Menyetujui,\\ \end{centering} \vspace{0.75cm}
\begin{minipage}[b]{0.45\linewidth}
% \centering Bandung, \makebox[0.5cm]{\hrulefill}/\makebox[0.5cm]{\hrulefill}/2013 \\
\vspace{2cm} \begin{center}
    Nama: \pembA \\ Pembimbing Utama
\end{center} 
\end{minipage} \hspace{0.5cm}
\begin{minipage}[b]{0.45\linewidth}
% \centering Bandung, \makebox[0.5cm]{\hrulefill}/\makebox[0.5cm]{\hrulefill}/2013\\
% \vspace{2cm} Nama: \pembB \\ Pembimbing Pendamping
\end{minipage}
\vspace{0.5cm}
}{
% \centering Bandung, \makebox[0.5cm]{\hrulefill}/\makebox[0.5cm]{\hrulefill}/2013\\
% \vspace{2cm} Nama: \pembA \\ Pembimbing Tunggal
}
\end{document}

